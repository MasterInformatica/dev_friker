% Created 2015-12-15 mar 19:17
\documentclass[11pt]{article}
\usepackage[utf8]{inputenc}
\usepackage[T1]{fontenc}
\usepackage{fixltx2e}
\usepackage{graphicx}
\usepackage{longtable}
\usepackage{float}
\usepackage{wrapfig}
\usepackage{soul}
\usepackage{textcomp}
\usepackage{marvosym}
\usepackage{wasysym}
\usepackage{latexsym}
\usepackage{amssymb}
\usepackage{hyperref}
\tolerance=1000
\providecommand{\alert}[1]{\textbf{#1}}

\title{BlockPuzzle, Propuesta para Práctica Final DEV}
\author{Doménech Arellano, Jesús Javier}
\date{15 de Diciembre de 2015}
\hypersetup{
  pdfkeywords={},
  pdfsubject={},
  pdfcreator={Emacs Org-mode version 7.9.3f}}

\begin{document}

\maketitle

\setcounter{tocdepth}{3}
\tableofcontents
\vspace*{1cm}

\section{Propuesta}
\label{sec-1}
\subsection{Descripción}
\label{sec-1-1}


  El juego que he elegido desarrollar, lo he bautizado como
  \textbf{BlockPuzzle}. Su funcionamiento es muy sencillo aunque
  superar niveles a veces puede volverse realmente complicado.\\

  El juego consiste en mover tu pieza (cubo o paralelepípedo de base
  cuadrada) por un mapa, tu figura irá girando por sus caras
  avanzando. El objetivo final es llevar a tu figura a una casilla
  especial, pero debes encajar en ella.
\subsection{Componentes}
\label{sec-1-2}

   El juego básicamente tendrá un par de escenas:
\begin{itemize}
\item Gestor de niveles.
\item El nivel (será una escena genérica que cargue un nivel a partir
     de un mapa).
\item HighScores.
\end{itemize}
   Además tendrá objetos que controlar, que son:
\begin{itemize}
\item Mapa.
\item Pieza que juega.
\item Puntuacion (por pasos y tiempo).
\item Niveles disponibles (y con suerte, posibilidad de niveles
     bloqueados).
\end{itemize}
\subsection{Características}
\label{sec-1-3}

  Inicialmente el objetivo de la práctica es conseguir una versión
  jugable con los siguientes detalles:
\begin{itemize}
\item Lee mapas de XML.
\item La figura tendrá una cara especial que será la que debe casar con
    la casilla objetivo.
\item Te puedes caer por los bordes o agujeros del mapa.
\item Al llegar a la casilla objetivo se pasa el nivel.
\item Se cuenta el numero de pasos, hay tiempo limite por nivel.
\item Acepta jugar con un Cubo o un paralelepípedo.
\item Permite mapas de diferentes alturas.
\item Memoria de Records de usuario.
\end{itemize}

\end{document}
